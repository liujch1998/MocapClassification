\documentclass{article}

\begin{document}

\textbf{Methodology} \\

Neural network is known to be one of the most well-performing architecture for machine learning. Its effectiveness is supported by numerous real-world applications. Neural network offers high expressivity for non-linear threshold functions. In our project, since different classes of movement are not necessarily linearly separable with respect to their features from sensor data, we consider adopting neural network architecture. \\

Feedforward neural network is the primitive setting, but neural network has many derivations, one of which is recurrent neural network (RNN). RNN takes states from previous time frame(s) as input in addition to the regular input at the current time frame, and is especially good at learning from features with sequential properties and does not require fixed input size. We will experiment with RNN and long short-term memory (LSTM), a derived architecture from RNN, and compare the results with those produced by feedforward neural network and other learning architectures. 

\end{document}
